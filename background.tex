\chapter{Background}
\label{ch:background}

\section{Text Classification}

\section{Attention models}
Attention models are an important concept in neural networks that has
been researched within diverse application domains.

\section{Existing works on \ac{icdo}}
Early works for ICD-O3 code assignment were based on rule-based
systems, where the code was assigned by creating a set of handcrafted
text search queries and combining results by standard Boolean
operators~\cite{crocetti_automatic_2004}. In order to prevent spurious
matches, rules need to be very specific, making it very difficult to
achieve a sufficiently high recall on future (unseen) cases.

A number of studies reporting on the application of machine learning
to this problem have been published during the last decade. Direct
comparisons among these works are impossible due to the (not
surprising) lack of a standard publicly available dataset and
heterogeneous details in the settings. Still, we highlight in the
following the main differences among them in order to provide some
background. In~\cite{jouhet_automated_2011}, the authors employed
support vector machine (SVM) and Naive Bayes classifiers on a small
dataset of $5\,121$ French pathology reports and a reduced number of
target classes (26 topographic classes and 18 morphological classes),
reporting an accuracy of 72.6\% on topography and 86.4\% on morphology
with SVM. A much larger dataset of $56\,426$ English reports from the
Kentucky Cancer Registry was later employed
in~\cite{kavuluru_automatic_2013}, where linear classifiers (SVM,
Naive Bayes, and logistic regression) were also compared but only on
the topography task and using 14, 42, and 57 classes. The authors
reported a micro-averaged F1 measure of 90\% when using SVM with both
unigrams and bigrams. Still, the bag-of-words representations used by
these linear classifiers do not take word order into account and are
unable to capture similarities and relations among words (which are
all represented by orthogonal vectors). Deep learning techniques are
known to overcome these limitations but were not employed to this
problem until very recently. In~\cite{qiu_deep_2018}, a convolutional
neural network (CNN) architecture fed by word vectors pretrained on
PubMed was applied to a small corpus of 942 breast and lung cancer
English reports with 12 topography classes; the authors demonstrate
the superiority of this approach compared to linear classifiers with
significant increases in both micro and macro F1
measure. In~\cite{gao_hierarchical_2018}, the same research group also
experimented on the same dataset using recurrent neural networks (RNN)
with hierarchical attention~\cite{yang_hierarchical_2016}, obtaining
further improvements over the CNN architecture.

%%% Local Variables:
%%% mode: latex
%%% TeX-master: "thesis"
%%% End:
