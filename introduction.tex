\chapter{Introduction}
\label{ch:introduction}

\section{Cancer registries}
Cancer is a major concern worldwide, as it decreases the quality of
life and leads to premature mortality. In addition it is one of the
most complex and difficult-to-treat
diseases, with significant social implications, both in terms of
mortality rate and in terms of costs associated with treatment and
disability~\cite{sullivan_delivering_2011,b._stewart_world_2014,desantis_cancer_2014,siegel_cancer_2016}.
Measuring the burden of disease is one of the main concerns of public
healthcare operators. Suitable measures are necessary to describe the general state
of population’s health, to establish public health goals and to
compare the national health status and performance of health systems
across countries. Furthermore, such studies are needed to assess the
allocation of health care and health research resources across disease
categories and to evaluate the potential costs and benefits of public
health interventions~\cite{brown_burden_2001}.

Cancer registries emerged during the last few decades as a strategic
tool to quantify the impact of the disease and to provide analytic
data to healthcare operators and decision makers.  Cancer registries
use administrative and clinical data sources in order to identify all
the new cancer diagnoses in a specific area and time period and
collect incidence records that provide details on the diagnosis and
the outcome of treatments.  Mining cancer registry datasets can help
towards the development of global surveillance
programs~\cite{tourassi_deep_2017} and can provide important insights
such as survivability~\cite{delen_predicting_2005}.  Although data
analysis software would best operate on structured representations of
the reports, pathologists normally enter data items as free text in
the local country language. This requires intelligent algorithms for
medical document information extraction, retrieval, and
classification, an area that has received significant attention in the
last few years (see, e.g.,~\cite{mujtaba_clinical_2019} for a recent
account and \cite{yim_natural_2016} for the specific case of cancer).

\section{\ac{icdo}}
Pathology reports can be classified according
to codes defined in the \ac{icdo3}
system~\cite{fritz_international_2000}, a specialization of the ICD
for the cancer domain which is internationally adopted as the standard
classification for topography and
morphology~\cite{airtum_handbook_2008}.  The development of text
analysis tools specifically devoted to the automatic classification of
incidence records according to ICD-O3 codes has been addressed in a
number of previous papers. These
works, however, have either focused on reasonably large datasets but
using simple linear classifiers based on bag-of-words representations
of text~\cite{jouhet_automated_2011,kavuluru_automatic_2013}, or have
applied recent state-of-the-art deep learning
techniques~\cite{gao_hierarchical_2018,qiu_deep_2018} but using
smaller datasets and restricted to a partial set of
tumors. Additionally, the use of deep learning techniques usually
requires accurate domain-specific word vectors (embeddings of words in
a vector space) that can be derived from word co-occurrences in large
corpora of unlabeled
text~\cite{mikolov_linguistic_2013,pennington_glove:_2014,devlin2018bert}. Large
medical corpora are easily available for English (e.g. PubMed) but not
necessarily for other languages.

A topographical \ac{icdo3} code is structured as \emph{Cmm.s} where
\emph{mm} and \emph{s} represent the main site and the subsite,
respectively. For example, \emph{C50.2}
is the code for the upper-inner quadrant (\emph{2}) of breast (\emph{50}).

A morphological \ac{icdo3} code is structured as \emph{tttt/b}
where \emph{tttt} and \emph{b} represent the cell type and the tumor
behavior (benign, uncertain, in-situ, malignant primary site,
malignant metastatic site), respectively. For example, \emph{8140/3}
is the code for an adenocarcinoma (\emph{adeno 8140};
\emph{carcinoma 3}).


%%% Local Variables:
%%% mode: latex
%%% TeX-master: "thesis"
%%% End:
