\chapter{Introduction}
\label{ch:introduction}

\section{Cancer registries}
Cancer is a major concern worldwide, as it decreases the quality of
life and leads to premature mortality. In addition it is one of the
most complex and difficult-to-treat
diseases, with significant social implications, both in terms of
mortality rate and in terms of costs associated with treatment and
disability~\cite{sullivan_delivering_2011,b._stewart_world_2014,desantis_cancer_2014,siegel_cancer_2016}.
Measuring the burden of disease is one of the main concerns of public
healthcare operators. Suitable measures are necessary to describe the general state
of population’s health, to establish public health goals and to
compare the national health status and performance of health systems
across countries. Furthermore, such studies are needed to assess the
allocation of health care and health research resources across disease
categories and to evaluate the potential costs and benefits of public
health interventions~\cite{brown_burden_2001}.

Cancer registries emerged during the last few decades as a strategic
tool to quantify the impact of the disease and to provide analytic
data to healthcare operators and decision makers.  Cancer registries
use administrative and clinical data sources in order to identify all
the new cancer diagnoses in a specific area and time period and
collect incidence records that provide details on the diagnosis and
the outcome of treatments.  Mining cancer registry datasets can help
towards the development of global surveillance
programs~\cite{tourassi_deep_2017} and can provide important insights
such as survivability~\cite{delen_predicting_2005}.  Although data
analysis software would best operate on structured representations of
the reports, pathologists normally enter data items as free text in
the local country language. This requires intelligent algorithms for
medical document information extraction, retrieval, and
classification, an area that has received significant attention in the
last few years (see, e.g.,~\cite{mujtaba_clinical_2019} for a recent
account and \cite{yim_natural_2016} for the specific case of cancer).

\section{\ac{icdo}}\label{sec:icdoCodes}
Pathology reports can be classified according
to codes defined in the \ac{icdo3}
system~\cite{fritz_international_2000}, a specialization of the ICD
for the cancer domain which is internationally adopted as the standard
classification for topography and
morphology~\cite{airtum_handbook_2008}.  The development of text
analysis tools specifically devoted to the automatic classification of
incidence records according to ICD-O3 codes has been addressed in a
number of previous papers. However, these
works have either focused on reasonably large datasets but
using simple linear classifiers based on bag-of-words representations
of text~\cite{jouhet_automated_2011,kavuluru_automatic_2013}, or
applied recent state-of-the-art deep learning
techniques~\cite{gao_hierarchical_2018,qiu_deep_2018} but using
smaller datasets and restricted to a partial set of
tumors. Additionally, the use of deep learning techniques usually
requires accurate domain-specific word vectors (embeddings of words in
a vector space) that can be derived from word co-occurrences in large
corpora of unlabeled
text~\cite{mikolov_linguistic_2013,pennington_glove:_2014,devlin2018bert}. Large
medical corpora are easily available for English (e.g. PubMed) but not
 for other languages.

A topographical \ac{icdo3} code is structured as \emph{Cmm.s} where
\emph{mm} represent the main site and \emph{s} the subsite. For example, \emph{C50.2}
is the code for the upper-inner quadrant (\emph{2}) of breast (\emph{50}).

A morphological \ac{icdo3} code is structured as \emph{tttt/b}
where \emph{tttt} represent the cell type and \emph{b} the tumor
behavior (benign, uncertain, in-situ, malignant primary site,
malignant metastatic site). For example, \emph{8140/3}
is the code for an adenocarcinoma (\emph{adeno 8140};
\emph{carcinoma 3}).

\section{Existing works on \ac{icdo}}
Early works for ICD-O3 code assignment were structured on rule-based
systems, where the code was assigned by creating a set of handcrafted
text search queries and combining results by standard Boolean
operators~\cite{crocetti_automatic_2004}. In order to prevent spurious
matches, rules need to be very specific, making it very difficult to
achieve a sufficiently high recall on future (unseen) cases.

A number of studies reporting on the application of machine learning
to this problem have been published during the last decade. Direct
comparisons among these works are impossible due to the (not
surprising) lack of standard publicly available datasets and the
presence of 
heterogeneous details in the settings. Still, we highlight the main
differences among them in order to provide some 
background. In~\cite{jouhet_automated_2011}, the authors employed
support vector machine (SVM) and Naive Bayes classifiers on a small
dataset of $5\,121$ French pathology reports and a reduced number of
target classes (26 topographic classes and 18 morphological classes),
reporting an accuracy of 72.6\% on topography and 86.4\% on morphology
with SVM. A much larger dataset of $56\,426$ English reports from the
Kentucky Cancer Registry was later employed
in~\cite{kavuluru_automatic_2013}, where linear classifiers (SVM,
Naive Bayes, and logistic regression) were also compared but only on
the topography task and using 14, 42, and 57 classes. The authors
reported a micro-averaged F1 measure of 90\% when using SVM with both
unigrams and bigrams. Still, the bag-of-words representations used by
these linear classifiers do not consider word order and are
unable to capture similarities and relations among words (which are
all represented by orthogonal vectors). Deep learning techniques are
known to overcome these limitations but were not applied to this
problem until very recently. In~\cite{qiu_deep_2018}, a \ac{cnn}
architecture fed by word vectors pretrained on 
PubMed was applied to a small corpus of 942 breast and lung cancer
reports in English with 12 topography classes; the authors demonstrate
the superiority of this approach compared to linear classifiers with
significant increases in both micro and macro F1
measures. In~\cite{gao_hierarchical_2018}, the same research group also
experimented on the same dataset using \acp{rnn}
with hierarchical attention~\cite{yang_hierarchical_2016}, obtaining
further improvements over the CNN
architecture. Also~\cite{gao2019classifying} is from the same group,
and in this work they overcome the problem of the small corpus and
experimented a new hierarchical self attention mechanism on a big
dataset of $374,899$ pathology reports obtained from the Louisiana
cancer register with 70 labels for site, 7 for laterality, 4 for
behavior, 516 for histology and 9 for grade. This latter work was
published after the writing of the present thesis and before the
public defence, thus we cited it in this section for completeness, but
we did not compare our work with theirs.

\section{Motivation and contributions}\label{sec:motivation}
With this work we want to answer to some questions.
\begin{description}
\item[Q1] At the best of our
  knowledge, there are no large scale studies on deep learning method
  applied to pathology reports. Cancer registries collect large amount of
  records and invest time labeling them with topographical and
  morphological codes. Existing works focus on \ac{icdo} classification
  of small datasets or using few 
  classes. We want to make a step further applying deep learning
  techniques to a large scale dataset.
\item[Q2] We want to apply novel deep learning techniques, like
  attention models and \ac{bert}, to cancer pathology reports text
  classification. We want to propose
  experiments to assess how they perform on this kind of data.
\item[Q3] It is not yet clear from the literature if, in this domain,
  bag-of-words techniques are inferior to deep learning
  techniques. The structure of the sentences in italian-language 
  cancer pathology reports differs substantially from the structure of the text
  where \ac{rnn} methods are usually employed. The records often
  do not present the usual subject-verb-object structure.
\item[Q4] It is not clear if the use of novel attention-based and
  hierarchical techniques 
  represents an improvement with respect to simpler \ac{rnn} methods on this
  kind of text. 
\item[Q5] It is not clear the contribution and the applicability of
  unsupervised learning techniques to uncommon text corpora. Usually,
  word vectors are trained on long english text corpora, taken from
  Wikipedia, Gigaword, or similar. Our documents not only are in a
  different language, but they also have a different structure.
\item[Q6] We want to evaluate the possibility to give
  some interpretation to the notoriously uninterpretable deep learning
  models. 
\end{description}


%%% Local Variables:
%%% mode: latex
%%% TeX-master: "thesis"
%%% End:
