\chapter{Conclusions}
Since the cancer registration process is partially based on manual
revision, including also the interpretation of the free text in
pathological reports, significant delays in data production and
publication may occur. This weakens data relevance for the purpose of
assessing compliance with updated regional recommended integrated case
pathways, as well as for public health purposes. Improving automated
methods to generate a list of putative incident cases and to
automatically estimate process indicators is thus an opportunity to
perform an up-to-date evaluation of cancer-care quality. In
particular, machine learning techniques like the ones presented in
this paper could overcome the delay in cancer case definition by the
cancer registry and allow a powerful tool for timely indicators
computation. The implementation of this procedure could guarantee an
automated and validated instrument to monitor and evaluate diagnostic
and therapeutic pathways.

We analyzed the available data and created different models in order
to implement an automated classification system. We obtained very
encouraging results in classifying cancer cases based on the
interpretation of free text in the data-flow of pathology
reports. This suggests that machine learning methods can be usefully
leveraged in this context.  We have also shown that unlabeled data can
be effectively used to construct useful word vectors and improve
classification accuracy. 

We also took advantage from the unlabeled data in order to improve
the classification. Ours models have also the added value that can be
utilized to retrieve records adjusting the precision-recall trade-off.

The use of administrative data sources that are up to date combined
with powerful machine learning techniques to automate text
classification is in the interest of the development of a standardized
surveillance system at Regional and National level. Stakeholders and
decision makers need timely and updated indicators to evaluate and
plan healthcare activities. The availability of timely indicators,
routinely and automatically produced, is technically possible. The
main novelty of this work is to show the power of machine learning
techniques applied to the classification of free text pathological
records. This was not yet been systematically implemented in other
Italian cancer registries. This provides a useful monitor tool for
cancer patients pathways, allowing to describe population’s general
health state and to establish public health goals.

The results of the interpretable models can be used to
assist the human classification process on simple records. It can be
used as
a form of text compression, highlighting the most important terms. On
more complex records it can be used to leverage the knowledge of the
model to gain insight on the decision process. To overcome the
limitations of the interpretable
model respect to the general model, in terms of
classification metrics, is it possible to combine the two variants. The
general model can be used to give a more
authoritative classification on the samples while at the same time,
the interpretable model can highlight the same samples.


%%% Local Variables:
%%% mode: latex
%%% TeX-master: "thesis"
%%% End:
